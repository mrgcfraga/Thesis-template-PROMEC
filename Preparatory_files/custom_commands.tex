%----------
% Commands
%----------
\newcommand{\ud}{\,\mathit{d}}												%Infinitesimal symbol for integration
\newcommand{\deriv}{\mathit{d}} 												%Derivation symbol
\newcommand{\vect}[1]{\mathbf{#1}}											%Bold latin letters to represent vectors
\newcommand{\gvect}[1]{\boldsymbol{\mathbf{#1}}}								%Bold greek letters to represent vectors
\newcommand{\uvec}[1]{\boldsymbol{\hat{\textbf{#1}}}}						%Symbol for unit vectors
\newcommand{\Parent}[1]{\left(#1\right)}										%Large parenthesis
\newcommand{\parent}[1]{(#1)}												%Small parenthesis
\newcommand{\Sqbrackets}[1]{\left[#1\right]}									%Large square brackets
\newcommand{\sqbrackets}[1]{[#1]}												%Small square brackets
\newcommand{\Chaves}[1]{\left\{#1\right\}}									%Large curly brackets
\newcommand{\chaves}[1]{\{#1\}]}												%Small curly brackets
\newcommand{\centertext}[2]{\parbox{#1}{\centering\nohyphens{#2}}}			%Center text in table cells
\newcommand{\Modulo}[1]{\left\vert{#1}\right\vert} 							%Large module symbol
\newcommand{\modulo}[1]{\vert{#1}\vert} 										%Small module symbol
\newcommand{\maximum}{\mathrm{max}}											%Maximum symbol
\newcommand{\lesfilt}[1]{\overline{#1}}										%Spatial filtering symbol
\newcommand{\lesflut}[1]{{#1}''}												%Symbol for fluctuations from spatial filtering
\newcommand{\flut}[1]{{#1}'}													%Symbol for fluctuations from the mean
\newcommand{\fl}[1]{\overset{\scriptscriptstyle\mathit{FL}}{\widehat{#1}}}	%Flux limiter symbol
\newcommand{\subscript}[1]{{(.)}_{#1}} 										%Subscript notation
\newcommand{\superscript}[1]{{(.)}^{#1}} 									%Superscript notation
\newcommand{\rref}[1]{Reaction~\ref{#1}}									%For referencing reactions
\newcommand{\transpose}[1]{{\bqty*{#1}}^{\!\mathsf{T}}}								%Symbol for matrix transposition
\newcommand{\concentration}[1]{[{#1}]}										%Volumetric concentration symbol
\newcommand{\Med}[1]{\left\langle{#1}\right\rangle}							%Large mean symbol
\newcommand{\med}[1]{\langle{#1}\rangle}										%Small mean symbol 
\newcommand{\subheadingspace}{\hspace{12pt}} 								%Indent of table subheadings
\newcommand{\conf}[1]{cf.~{#1}}												%Confer symbol
\newcommand{\mesh}[2]{$\text{Q}_{#1}{#2}$}
\newcommand{\vecmag}[1]{\left\vert\left\vert{#1}\right\vert\right\vert}
\newcommand{\tri}{\text{TRI}}												%Symbol for the "with TRI" solution
\newcommand{\ntri}{\text{nTRI}}												%Symbol for the "without TRI" solution
\newcommand{\glosunit}[1]{, {#1}} 											%To format glossary units
\newcommand{\aref}[1]{Appendix~\ref{#1}}										%Cross-referencing annexes
\newcommand\reallywidetilde[1]{\ThisStyle{%
	\setbox0=\hbox{$\SavedStyle#1$}%
	\stackengine{1\LMpt}{$\SavedStyle#1$}{%
		\stretchto{\scaleto{\SavedStyle\mkern.2mu\sim}{.5467\wd0}}{.4\ht0}%
		%    .2mu is the kern imbalance when clipping white space
		%    .5467++++ is \ht/[kerned \wd] aspect ratio for \sim glyph
	}{O}{c}{F}{T}{S}%
	}}
\newcommand{\favre}[1]{\reallywidetilde{#1}}								%Favre filtering symbol
\newcommand{\Favre}[1]{\reallywidetilde{#1}}								%Larger favre filtering symbol
\newcommand{\dynfilt}[1]{\widehat{#1}}
\newcolumntype{L}[1]{>{\raggedright\let\newline\\\arraybackslash\hspace{0pt}}p{#1}}
\newcolumntype{C}[1]{>{\centering\let\newline\\\arraybackslash\hspace{0pt}}p{#1}}
\newcolumntype{R}[1]{>{\raggedleft\let\newline\\\arraybackslash\hspace{0pt}}p{#1}}
\newcommand{\prob}[1]{\text{Prob}\Bqty{#1}} %Probability
\newcommand{\sgsstress}[2]{\tau^{\mathit{sgs}}_{#1#2}}
\newcommand{\sr}[1]{S_{#1}}
\newcommand{\shs}{\hspace{1em}}
\newcommand{\nR}[1]{R^{\star}_{#1}}		%Normalized correlation coefficient
\newcommand{\nU}[1]{U^{\star}_{#1}}			%Normalized unresolved group
\newcommand{\app}[2]{{#1}^{\text{ap}}_{#2}} %Approximated quantity
\newcommand{\hsp}{\hspace{1pt}}

\def\PP{\rule{0pt}{1.5ex}P}

%%----------------
%%	ENVIRONMENTS
%%----------------
\newcounter{chem}[chapter]
\newcounter{temp}
\newenvironment{chequation}{%
	\setcounter{temp}{\value{equation}}%
	\setcounter{equation}{\value{chem}}%
	\renewcommand{\theequation}{R\thechapter.\arabic{equation}}%
}{%
\setcounter{chem}{\value{equation}}%
\setcounter{equation}{\value{temp}}%
}

%%-------------
%%	NEW UNITS
%%-------------
\DeclareSIPostPower\tothefourth{4}											%Fourth power as a unit
\DeclareSIPostPower\tothefifth{5}											%Fifth power as a unit
\DeclareSIUnit\atm{atm}														%atm unit

%%-------------
%%	SAVEBOXES
%%-------------
\newsavebox{\largestimage}