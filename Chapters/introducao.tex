\chapter{INTRODUÇÃO}\label{Cap:introducao}
Este modelo de editoração de teses e dissertações tem por finalidade padronizar a produção científica realizada, em nível acadêmico, pelos alunos do Programa de Pós-Graduação em Engenharia Mecânica, PROMEC. As condições de formatação aqui apresentadas devem ser rigorosamente seguidas. Contudo, a divisão e escolha dos capítulos será ditada pelo próprio trabalho realizado e fica a cargo do aluno. Os exemplos de seções aqui expostos servem como um possível modelo a ser adotado, mas não o único e (possivelmente) não o melhor.
 
O capítulo introdutório do texto é responsável por apresentar ao leitor o contexto em que o trabalho se enquadra. Sugere-se que sejam apresentadas as circunstâncias e razões que levaram ao projeto em questão. De maneira simplista, pode-se enumerar os três principais itens abordados na Introdução:
\begin{itemize}
\item A realidade em que o problema se encontra (o quê?).
\item O propósito em se abordar determinado problema (por quê?).
\item O escopo da pesquisa desenvolvida (como?).
\end{itemize}

A Introdução não deve ser confundida com as seções de Revisão Bibliográfica e Fundamentação Teórica. Na seção introdutória, deve-se apresentar o trabalho em questão expondo a relevância do mesmo.

\section{Revisão Bibliográfica}
Esta seção tem por objetivo apresentar o estado da arte quanto aos principais temas abordados na tese/dissertação. Devem ser relacionadas as contribuições mais significativas no campo de estudos no qual o trabalho se insere. Aconselha-se que os trabalhos que precedem ou são utilizados como base para a tese/dissertação venham a ser tratados de forma mais detalhada.

\section{Objetivos}
Normalmente, após a introdução ao problema a ser abordado, são enumerados os objetivos a serem buscados. Comumente, um objetivo principal (final) é definido associado a objetivos secundários (intermediários). Note que os objetivos podem ser apresentados na forma de texto, listados como \textit{bullet points} usando o ambiente \texttt{itemize} (como ilustrado na seção anterior), ou inumerados usando o ambiente \texttt{enumerate}, como exemplificado a seguir.
\begin{enumerate}
    \item Primeiro objetivo
    \item Segundo objetivo
    \item Etc.
\end{enumerate}

\section{Organização do Trabalho}
Trabalhos científicos são, geralmente, extensos. Desde modo, é comum a exposição de uma seção na qual a estrutura de organização do trabalho é apresentada de forma sucinta. Normalmente, essa seção lista os capítulos contidos no texto, bem como uma breve descrição de cada um. Esta seção em particular se beneficia do uso de referências cruzadas---por exemplo, ``o \cref{Cap:introducao} apresenta uma introdução ao texto; em seguida, o \cref{Cap:fundamentacao-teorica} traz a fundamentação teórica do trabalho\ldots'' Referências cruzadas podem ser implementadas em \LaTeX\ de forma muito conveniente usando o pacote \texttt{cleveref}. Isso é brevemente discutido na \cref{Sec:referencias-cruzadas}.


