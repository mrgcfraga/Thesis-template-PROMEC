\chapter{REGRAS DE FORMATAÇÃO}\label{Cap:fundamentacao-teorica}
A seguir são apresentadas as disposições gerais quanto a formatação de teses e dissertações. O presente modelo foi construído empregando a linguagem {\LaTeX}. O leitor perceberá que, para propiciar uma melhor compreensão da sintaxe empregada, fragmentos do código fonte utilizado são ocasionalmente apresentados. Mais detalhes podem evidentemente ser encontrados consultando os arquivos \texttt{.tex} incluídos.

A distribuição \textit{{\TeX} Live 2016} \citep{rahtz2016livetex} foi empregada para a preparação deste documento. Qualquer editor {\TeX} deve ser compatível com este modelo. Contudo, somente os editores \textit{{\TeX}Maker} (v.4.5 \& Qt4) e \textit{{\TeX}Studio} (v.2.12.2 \& Qt5) foram testados. Recomenda-se para um usuário inexperiente em \LaTeX\ elabore seu documento usando o \textit{website} \href{https://www.overleaf.com}{Overleaf}.

\section{Recomendações Gerais}
As teses e dissertações devem empregar a fonte serifada Times New Roman. Em \LaTeX, versões dessa fonte para texto e matemática estão disponíveis através dos pacotes \texttt{newtxtext} e \texttt{newtxmath}, respectivamente (já incluídos na classe \texttt{promec.cls}). O tamanho de fonte deve ser definido como \SI{12}{pt} com um espaçamento entre linhas de \num{1.5}. As margens do texto devem ser estar de acordo com as normas da ABNT, conforme listado na \cref{Tab:margens}.
\begin{table}[tb]
    \centering
    \caption{Margens do documento}
    \begin{tabular}{cccc}
        \toprule
        Margem superior & Margem inferior & Margem esquerda & Margem direita  \\
        \midrule
        \SI{3}{\centi\meter} & \SI{2}{\centi\meter} & \SI{3}{\centi\meter} & \SI{2}{\centi\meter} \\
        \bottomrule
    \end{tabular}
    \label{Tab:margens}
\end{table}

Ao longo do texto será usado o estilo simples, reservando-se o \textbf{negrito} para os títulos e subtítulos e o \textit{itálico} para realce de novos termos ou de termos em língua estrangeira. Deve-se empregar papel em tamanho A4, utilizando apenas um lado da folha. O texto deve ser escrito em linguagem impessoal de acordo com a norma culta e em conformidade com a reforma ortográfica da língua portuguesa de 2009 \citep{bechara2012moderna}.

\section{Paginação}
Todas as páginas pré-textuais devem ser contadas para numeração (incluindo capa). Contudo, somente a partir da contra-capa (inclusa) se inicia a numeração das páginas pré-textuais. Essa numeração deve empregar algarismos romanos em caixa baixa e centralizados na base da página. As páginas referentes ao corpo do texto devem ser numeradas utilizando algarismos arábicos, em sequência progressiva crescente, excluindo-se todas as páginas pré-textuais. A numeração deve estar localizada no canto superior direito a \SI{2}{\centi\meter} da borda da página. Normalmente, a primeira página textual a ser numerada é a da Introdução, a qual inicia a contagem no número \num{1} (um). Anexos e/ou apêndices devem ser numerados em continuidade as páginas textuais, sem reiniciar a contagem.

Controle da paginação é feito automaticamente pela classe \texttt{promec.cls}. Dois comandos são disponibilizados nesse sentido: \lstinline!\frontmatter! e \lstinline!\mainmatter!, que devem ser posicionados antes do início dos elementos pré-textuais e antes do corpo do texto, respectivamente. Nos arquivos \texttt{.tex} de exemplo, esses comandos já estão incluídos (ver em \texttt{Pretext/pretext.tex}).

Os capítulos devem ser numerados utilizando algarismos arábicos. Suas seções e subseções seguem a mesma sequência de numeração, empregando pontos (.) como separador. Os títulos dos capítulos devem ser escritos em CAIXA ALTA e em \textbf{negrito}. Títulos de seções e sub-seções também empregam \textbf{negrito}, porém somente a primeira letra de cada palavra é escrita em CAIXA ALTA. Note que todos esses pontos são resolvidos automaticamente pela classe \texttt{promec.cls}, \textbf{exceto} a capitalização dos títulos dos capítulos e seções; portanto, o usuário deve garantir que os títulos estejam capitalizado (isto é, deve-se escrever ``INTRODUÇÃO'' e não ``Introdução'').

\section{Citações}
A citação de trabalhos científicos deve ser realizada de modo a referenciar o autor (fonte) de determinada afirmação (fato, experimento, teoria, ...). Citações devem ser compreendidas no corpo do texto, podem ser realizadas de forma direta ou indireta. Não deve-se usar notas de rodapé\footnote{Notas de rodapé devem apresentar informações de fim meramente complementar. Essas notas são posicionados na margem inferior da página onde ocorrem.} para esta finalidade.

\subsection{Citações Diretas}
As citações diretas apresentam o nome do autor seguido pelo ano de sua respectiva publicação. Exemplos deste tipo de citação são enumerados a seguir:
\begin{enumerate}[label={\roman*.},ref={\roman*}]
	\item ``\ldots conforme apresentado por \citealp{dantzig1963linear}, a programação linear\ldots''
	\item ``\ldots de acordo com \citealp{haftka1991elements}, o método de otimização\ldots''
	\item ``Os resultados obtidos por \citealp{deleon2015stress} demonstram\ldots''\label{item:multiplos-autores}
\end{enumerate}
Quando o trabalho a ser referenciado possuir mais de dois autores (Item \ref{item:multiplos-autores}), a citação é realizada apresentando o nome do autor principal seguido da expressão “et alli” (ou abreviadamente “et al.”) e a data. Citações diretas podem ser obtidas empregando o comando \lstinline!\citealp{}!, onde o termo entre chaves deve apresentar o identificador de uma referência previamente contida no arquivo de bibliografias Bib\TeX. No presente modelo, o nome do arquivo contendo todas as referências é \texttt{bibliografia{\_}template.bib} e ele está localizado na pasta \texttt{Preparatory}. 

\subsection{Citações Indiretas}
Citações indiretas apresentam o autor de determinada referência sem fazer menção a ele explicitamente. Deste modo, a citação é realizada envolvendo por colchetes o nome do autor seguido pela data. Alguns exemplos dessa técnica são enumerados da seguinte forma:
\begin{enumerate}[label={\roman*.},ref={\roman*}]
	\item ``\ldots foi o trabalho que ajudou a cunhar o método através de sua aplicação em um problema de engenharia \citep{bathe1996finite}.''
	\item ``\ldots definido como o Método dos Elementos Finitos, MEF \citep{hughes2012finite}.''
	\item ``\ldots conforme apresentado por diferentes autores \citep{reddy2006introduction, zienkiewicz2005finite, bathe1996finite}.''
	\item ``\ldots essa formulação para contato \citep{man1993incremental} também é empregada por outros trabalhos presentes na literatura \citep{man1993engineering}.''\label{item:same_authors}
\end{enumerate}
Esse tipo de citação é obtido empregando o comando \lstinline!\citep{}!. Note a existência de dois trabalhos do mesmo autor(es) publicados no mesmo ano (Item \ref{item:same_authors}). Em casos como esse, deve-se proceder adicionando caracteres latinos (em ordem alfabética) diretamente após a data de cada citação. 

\section{Transcrições}
Em casos de citação de trechos parciais ou completos de algum trabalho, deve-se proceder transcrevendo o texto desejado em parágrafo próprio envolvendo o texto por aspas (`` ''). Nessas situações a referência pode ser feita de forma direta ou indireta. A seguinte transcrição\footnote{Tradução livre do texto originalmente publicado em Inglês.} de um pronunciamento do astronauta John Glenn\footnote{John Glenn foi o segundo astronauta a entrar em órbita terrestre. Este feito foi realizado em 1962 a bordo da cápsula espacial Friendship 7} exemplifica esse tipo de citação:
``Acho que a pergunta mais frequente que escuto é: `Quando você estava sentado naquela cápsula ouvindo a contagem regressiva, como você se sentiu?' Bem, a resposta é fácil. Eu me senti exatamente como você se sentiria caso estivesse se preparando para o lançamento sabendo que você está sentado em cima de dois milhões de peças---todas construídas pelo licitante que ofereceu o menor preço em um contrato do governo.'' \citep{kranz2009failure}

\section{Equações}
Equações devem ser centralizadas horizontalmente. Todas as equações que vierem a ser utilizadas/citadas posteriormente a sua aparição no texto devem ser numeradas. Essa numeração empregará algarismos arábicos em ordem crescente. O formato da numeração seguirá a sequência de numeração do capítulo em que a equação é apresentada e será compreendida entre parênteses, indexada como Capítulo.Equação. Por fazerem parte do texto, equações não devem simplesmente surgir ao acaso, elas precisam ser introduzidas no contexto no trabalho e estão sujeitas as regras de pontuação.

Como um primeiro exemplo, é apresentado o teorema de Pitágoras. Para um triângulo retângulo com catetos $a$ e $b$ e hipotenusa $c$, este teorema estabelece que
\begin{equation}
    a^2 + b^2 = c^2\,.
    \label{Eq:pitagoras}
\end{equation}
Esta equação foi produzida empregando a sintaxe
\begin{lstlisting}
\begin{equation}
    a^2 + b^2 = c^2\,.
    \label{Eq:pitagoras}
\end{equation}
\end{lstlisting}
e ela ilustra alguns pontos importantes de formatação e boas-práticas. Primeiramente, note que a \cref{Eq:pitagoras} faz parte da frase que começa ``Para um triângulo\ldots''; como pontuado anteriormente essa é a forma correta de se introduzir qualquer equação---citar a equação pelo seu número \textbf{antes} de ela ter sido introduzida não é correto. Como exemplo do que \textbf{não deve} ser feito, considere o próximo parágrafo sobre a fórmula quadrática.

A fórmula quadrática, ou fórmula de Bhaskara, permite encontrar as raízes de um polinômio de segunda ordem. Ela é dada pela \cref{Eq:bhaskara}.
\begin{equation}
    x = \frac{-b \pm \sqrt{b^2 - 4ac}}{2c}
    \label{Eq:bhaskara}
\end{equation}

Retornando à \cref{Eq:pitagoras}, como esta equação faz parte do texto, ela é pontuada. No caso, como a \cref{Eq:pitagoras} encerra uma frase, coloca-se um ponto final ao final da equação. Recomenda-se que um pequeno espaço horizontal seja inserido entre o fim da equação e o ponto final, para evitar que um leitor pense que o ponto final é um sinal que faz parte da equação. Tal espaço horizontal foi inserido através do comando \lstinline!\,!.

Além disso, todas variáveis presentes na \cref{Eq:pitagoras} foram definidas antes da equação ser introduzida. Não é necessário que as variáveis sejam definidas \textbf{antes} da equação, mas, caso isso não seja feito, elas devem ser definidas imediatamente após à equação (ver os próximos exemplos). É obrigatória que uma completa Lista de Símbolos esteja presente no texto, como um dos elementos pré-textuais. Note que nem todas as variáveis da \cref{Eq:bhaskara} foram definidas explicitamente; essa é mais uma instância na qual a apresentação dessa equação foi feita incorretamente.

Ocasionalmente pode ser necessário apresentar duas ou mais equações em sucessão. A forma recomendada de se fazer isso é com o ambiente \lstinline!gather!. Por exemplo, considere as definições de velocidade $V$ e aceleração $a$, dadas respectivamente como
\begin{gather}
    V = \dv{x}{t}\,, 
    \label{Eq:velocidade}\\[12pt]
    a = \dv{V}{t} = \dv[2]{x}{t}\,,
    \label{Eq:aceleracao}
\end{gather}
onde $x$ é a posição e $t$ é o tempo. Estas equações foram produzidas através da seguinte sintaxe:
\begin{lstlisting}
\begin{gather}
    V = \dv{x}{t}\,, 
    \label{Eq:velocidade}\\[12pt]
    a = \dv{V}{t} = \dv[2]{x}{t}\,,
    \label{Eq:aceleracao}
\end{gather}
\end{lstlisting}
O comando \lstinline!\\[12pt]! serve para quebrar a linha entre as duas equações e inserir um espaço vertical adequado entre elas (igual ao tamanho da fonte do texto). Note ainda que ambas as equações estão devidamente pontuadas.

Equações podem também ser apresentadas em grupos usando o ambiente \lstinline!subequations!. Como exemplo, considere as seguintes séries infinitas,
\begin{subequations}
    \begin{equation}
        \frac{\pi}{2} = \sum_{k=0}^{\infty} \frac{k!}{\pqty{2k + 1}!!}\,,
    \end{equation}
    \begin{equation}
        \pi = 3 + \frac{1^2}{6 + \frac{3^2}{6+\frac{5^2}{6+\frac{7^2}{6+\ddots}}}}\,.
    \end{equation}
\end{subequations}
Essas equações foram obtidas usando a sintaxe
\begin{lstlisting}
\begin{subequations}
    \begin{equation}
        \frac{\pi}{2} = \sum_{k=0}^{\infty} 
            \frac{k!}{\pqty{2k + 1}!!}\,,
    \end{equation}
    \begin{equation}
        \pi = 3 + \frac{1^2}{6 + 
            \frac{3^2}{6+\frac{5^2}{6+
                \frac{7^2}{6+\ddots}}}}\,.
    \end{equation}
\end{subequations}
\end{lstlisting}

Sempre que alguma grandeza for explicitada no texto ou em uma equação, as unidades devem ser apresentadas seguindo o Sistema Internacional (SI). Em casos excepcionais, nos quais se faz necessário o emprego de unidades em sistemas que não o SI, a grandeza equivalente no SI deve ser apresentada. Vide exemplo: ``\ldots foram empregadas placas metálicas com meia polegada de espessura (\SI{12.7}{\milli\meter}).''

\section{Figuras}\label{Sec:figuras}
Figuras devem ser apresentadas tão próximo quando possível da primeira posição em que elas são referenciadas. Não é necessário que elas apareçam após tal posição, desde que sejam mantidas na mesma página (nunca em uma página anterior, e nunca duas páginas após a posição de primeira referência). Caso o usuário deseje forçar que a figura apareça necessariamente após a posição de primeira referência, ele pode incluir o pacote \lstinline!flafter! no preâmbulo do documento.

Uma exemplo de figura é mostrado na \cref{Fig:figura_exemplo}.
\begin{figure}[tb]
    \centering
    \includegraphics{example-image-a}
    \caption{Exemplo de figura}
    \label{Fig:figura_exemplo}
\end{figure}
Todas as figuras devem possuir uma legenda, posicionada logo abaixo da figura e centralizada horizontalmente. Cada figura deve estar numerada seguindo indexação similar {a} das equações; a palavra ``Figura'' deve preceder o número e um travessão (--) deve ser colocado entre o número e o texto da legenda.

O parágrafo anterior foi gerado seguindo a sintaxe (alguns aspectos foram alterados em relação ao código-fonte original para facilitar a leitura)
\begin{lstlisting}[breaklines=true]
Uma exemplo de figura \'{e} mostrado na \cref{Fig:figura_exemplo}.
\begin{figure}[tb]
    \centering
    \includegraphics{example-image-a}
    \caption{Exemplo de figura}
    \label{Fig:figura_exemplo}
\end{figure}
Todas as figuras devem possuir uma legenda, posicionada logo abaixo da figura e centralizada horizontalmente. 
Cada figura deve estar numerada seguindo indexa\c{c}\~{a}o similar \`{a} das equa\c{c}\~{o}es; a palavra ``Figura'' deve preceder o n\'{u}mero e um travess\~{a}o (--) deve ser colocado entre o n\'{u}mero e o texto da legenda.
\end{lstlisting}
Note que:
\begin{itemize}
    \item O comando \lstinline!\centering! é incluído para garantir que a figura e a legenda fiquem centralizados
    \item O comando \lstinline!\caption{}! é colocado \textbf{após} a inclusão da figura (\lstinline!\includegraphis{}!); este é o correto para figuras
    \item O ambiente \lstinline!figure! tem como argumentos opcionais de posicionamento \lstinline![tb]!. É recomendado que estes sejam sempre os argumentos utilizados para \textit{floats} como figuras e tabelas, pois garantem o melhor posicionamento destes elementos (no topo ou no final da página). Não é recomendado utilizar-se o argumento \lstinline![h]!
    \item O ambiente \lstinline!figure! é inserido imediatamente após o final da frase em que a figura é primeiro citada, e não após o final do parágrafo. Isso assegura o melhor posicionamento da figura no PDF
\end{itemize}

Figuras compostas de subfiguras podem também ser incluídas utilizando o pacote \lstinline!subcaption! Um exemplo é mostrado na \cref{Fig:subfigura_exemplo}.
\begin{figure}[tb]
    \centering
    \begin{subfigure}[b]{0.4\textwidth}
        \centering
        \includegraphics[width=\textwidth]{example-image-b}
        \caption{Uma subfigura}
        \label{Fig:subfigura-a}
    \end{subfigure}
    \begin{subfigure}[b]{0.4\textwidth}
        \centering
        \includegraphics[width=\textwidth]{example-image-c}
        \caption{Outras subfigura}
        \label{Fig:subfigura-b}
    \end{subfigure}
    \caption{Exemplo de múltiplas figuras}
    \label{Fig:subfigura_exemplo}
\end{figure}
O segmento de código usado para gerar essas figuras é mostrado a seguir.
\begin{lstlisting}
\begin{figure}[tb]
    \centering
    \begin{subfigure}[b]{0.4\textwidth}
        \centering
        \includegraphics[width=\textwidth]{example-image-b}
        \caption{Uma subfigura}
        \label{Fig:subfigura-a}
    \end{subfigure}
    \begin{subfigure}[b]{0.4\textwidth}
        \centering
        \includegraphics[width=\textwidth]{example-image-c}
        \caption{Outras subfigura}
        \label{Fig:subfigura-b}
    \end{subfigure}
    \caption{Exemplo de m\'{u}ltiplas figuras}
    \label{Fig:subfigura_exemplo}
\end{figure}
\end{lstlisting}
Para mais instruções sobre o uso do ambiente \lstinline!subfigure!, o leitor é direcionado ao manual do pacote \lstinline!subcaption!.

Note que tanto as figuras principais (\cref{Fig:figura_exemplo,Fig:subfigura_exemplo}) como as subfiguras no exemplo anterior (\cref{Fig:subfigura-a,Fig:subfigura-b}) podem ser facilmente referenciadas desde que \lstinline!label{}!s tenham sido dados a cada uma delas. A forma correta de se realizar referenciação cruzada na classe \texttt{promec.cls} é discutida na \cref{Sec:referencias-cruzadas}.

\section{Tabelas}
Um exemplo de tabela é mostrado na \cref{Tab:exemplo}.
\begin{table}[tb]
    \centering
    \caption{Exemplo de tabela}
    \label{Tab:exemplo}
    \begin{tabular}{cc}
        \toprule
        Coluna 1 & Coluna 2 \\
        \midrule
        Dados 1 & Dados 2 \\
        Dados 3 & Dados 4 \\
        \bottomrule
    \end{tabular}
\end{table}
A sintaxe usada para produzir essa tabela é mostrada abaixo.
\begin{lstlisting}
\begin{table}[tb]
    \centering
    \caption{Exemplo de tabela}
    \label{Tab:exemplo}
    \begin{tabular}{cc}
        \toprule
        Coluna 1 & Coluna 2 \\
        \midrule
        Dados 1 & Dados 2 \\
        Dados 3 & Dados 4 \\
        \bottomrule
    \end{tabular}
\end{table}
\end{lstlisting}

As mesmas instruções e recomendações gerais dadas para o posicionamento e a formatação de figuras na \cref{Sec:figuras} são válidas também para tabelas. Uma diferença importante é que, para tabelas, a legenda deve vir \textbf{antes} da tabela, e não depois. Isso é obtido posicionando-se o comando \lstinline!\caption{}! antes da inserção do conteúdo da tabela.

Além disso, lista-se abaixo algumas recomendações de boas-práticas para a confecção de tabelas para trabalhos científicos.
\begin{itemize}
    \item Evite linhas verticais separando as colunas
    \item Evite também linhas horizontais separando as fileiras das tabelas, com exceção de uma linha no topo, uma ao final, e uma separando o cabeçalho do corpo da tabela. Sugere-se o uso dos comandos \lstinline!\toprule!, \lstinline!\bottomrule!, e \lstinline!\midrule! para produzir tais linhas, respectivamente. Esses comandos são parte do pacote \lstinline!booktabs!, que é incluído como parte da classe \texttt{promec.cls}
    \item Evite colocar bordas na tabela
    \item A fonte da tabela deve ser a mesma do corpo do texto
    \item Não coloque o texto no cabeçalho em negrito
    \item Na dúvida, alinhe o texto de todas as células à esquerda
    \item Inclua todas as unidades na tabela sempre que possível (e não na legenda)
\end{itemize}

\section{Referências Cruzadas}\label{Sec:referencias-cruzadas}
Referências cruzadas consistem em referências a outros elementos do texto dentro do próprio texto---por exemplo, ao \cref{Cap:introducao}, à \cref{Fig:figura_exemplo}, às \cref{Eq:pitagoras,Eq:bhaskara}, etc. Referências cruzadas são inevitáveis em textos acadêmicos. Na classe \texttt{promec.cls}, todos (ou quase todos, ver abaixo) os elementos podem ser referenciados através de um único comando, \lstinline!\cref{}!, parte do pacote \lstinline!cleveref!, que já é incluído na classe. Este comando permite referenciar um ou múltiplos elementos do texto---por exemplo, as referências feitas no início desse parágrafo foram geradas seguindo a seguinte sintaxe:
\begin{lstlisting}[breaklines=true]
por exemplo, ao \cref{Cap:introducao}, \'{a} \cref{Fig:figura_exemplo}, \'{a}s \cref{Eq:pitagoras,Eq:bhaskara}, etc
\end{lstlisting}

As referências cruzadas produzidas usando o comando \lstinline!\cref{}! estão automaticamente no formato exigido para as dissertações e teses do PROMEC. O comando também se adapta adequadamente aos idiomas inglês e português (a produção de documentos nestes dois idiomas é discutida na \cref{Sec:especificacao-classe}). 

A maioria dos elementos textuais podem ser referenciados usando o comando \lstinline!\cref{}!. No entanto, existem as seguintes exceções:
\begin{enumerate}
    \item Anexos
    \item Números de página
    \item Itens em listas
\end{enumerate}
Referências a estes elementos devem ser feitas da forma tradicional em \LaTeX, usando o comando \lstinline!\ref{}! antecedido pelo nome do elemento.

\section{Elementos pré-textuais}
Certos conteúdos devem ser obrigatoriamente inclusos antes do corpo principal do texto. Outros conteúdos pré-textuais são opcionais, e, caso o autor desejar, devem ser incluídos apenas na versão já aprovada do texto. Estes elementos são listados a seguir.
\begin{enumerate}
    \item Página título (obrigatória): caso o documento seja escrito em português, apenas uma página de título deve ser incluída. Caso o documento seja escrito em inglês, duas páginas de título devem ser incluídas: a primeira, com todos os textos, em inglês; e a segunda, em português. A(s) página(s) de título são as únicas páginas do documento que não devem ser numeradas
    \item Página de rosto (obrigatória), contendo informações importantes sobre o trabalho: a área de concentração, nome do orientador e coorientador, banca, nome do atual coordenador do PROMEC, entre outras
    \item Agradecimentos (opcional), devendo ser escrita seguindo a mesma formatação do texto no corpo do documento. Para pesquisadores bolsistas, um agradecimento à agência de fomento responsável pela bolsa é obrigatório
    \item Resumo em português (obrigatório), devendo ser escrito em um único parágrafo sem indentação, respeitando um máximo de \num{500} palavras. As palavras-chave do trabalho também são incluídas nesta página. O resumo, em conjunto com a lista de palavras-chave, não deve exceder uma página
    \item Resumo em inglês (obrigatório), ou \textit{Abstract}. Consiste na tradução livre do conteúdo no resumo, e as mesmas orientações gerais daquele elemento se aplicam a esse
    \item Índice (obrigatório), autoexplicativo
    \item Lista de figuras (obrigatório), autoexplicativo
    \item Lista de tabelas (obrigatório), autoexplicativo
    \item Lista de acrônimos e abreviações (obrigatório). Deve listar, em ordem alfabética, todos os acrônimos e abreviações usados no texto e dar os seus significados
    \item Lista de símbolos (obrigatório). Deve listar todos os símbolos usados no texto, incluindo subscritos e sobrescritos e símbolos para operadores (por exemplo, médias, tensores). Os símbolos devem estar separados em listas de acordo com as categorias (nesta ordem): símbolos latinos, símbolos gregos, subscritos, sobrescritos, operadores. Em cada categoria, os símbolos devem estar listados em ordem alfabética
\end{enumerate}
A maioria desses elementos são inseridos através de comandos e não precisam ser montados pelo usuário. As exceções são os elementos opcionais, os resumos, a listas de acrônimos e abreviações, e a lista de símbolo. Sugere-se que o usuário verifique o arquivo \texttt{Pretext/pretext.tex} para um melhor entendimento sobre a preparação de cada elemento.

\section{Elementos pós-textuais}
Imediatamente após o corpo principal do texto, a lista de referências à trabalhos anteriores deve ser incluída como um capítulo separado (não numerado). A impressão da lista de referências bibliográficas é gerenciada pelo próprio \LaTeX, sendo necessário apenas que o usuário prepare um arquivo \texttt{.bib} de forma adequada (ver o Apêndice~\ref{Cap:bib}).

Em seguida, opcionalmente pode-se incluir apêndices e anexos (nesta ordem). Estes elementos são incluídos com a mesma sintaxe de capítulos, e eles podem incluir seções e subseções. No entanto, apêndices são numerados por letras latinas maiúsculas, começando pelo Apêndice A; seções e subseções seguem a numeração padrão, então pode-se ter uma Seção A.1, uma Subseção A.1.1, etc. Anexos, por sua vez, são numerados números gregos maiúsculos, começando pelo Anexo I (e segue-se uma Seção I.1, uma subseção I.1.1, etc.).

A formatação de apêndices e anexos é feita no presente \textit{template} através dos comandos \lstinline!\appendix! e \lstinline!\annex!. O comando \lstinline!\appendix! deve ser colocado antes do primeiro apêndice (e apenas uma única vez no documento); o comando \lstinline!\annex!, antes do primeiro anexo, e também apenas uma única vez no documento.
